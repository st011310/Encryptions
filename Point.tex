\documentclass{article}

\usepackage[russian]{babel}
\usepackage[unicode]{hyperref}


\title{Повелитель проектов}

\begin{document}
\maketitle
\tableofcontents
\newpage
\section{Постановка}
	Нужно создать проект, который способен зашифровать и расшифорвать любое сообщение, если это возможно.
	\subsection{Реализация на Python}
		Python -- довольно медленный язык, поэтому планируется хранить сообщение и ключ в виде строки и отображать каждый шаг.
\section{Основы}
	\subsection{Ввходные данные}
	Сначала вводится название алгоритма.
	Затем ключевая буква <<E>> или <<D>> от \_ и \_ -- зашифровать или расшифоровать сообщение.
	Далее -- ключ.
	Всё остальное считать текстом.\\
\section{Алгоритмы}
	\subsection{DES}
	Реализован по \href{https://www.youtube.com/watch?v=Qn6WIgvBUw4}{\textit{видео-уроку}}.
	Внимание! У автора видео есть маленькая ошибка в получении промежуточного предложения после первого раунда.
	\subsection{Triple DES}
\section{Идеи и предложения}
	Использовать библиотеку \texttt{manim} в \texttt{Python} для анимации работы каждого из алгоритмов шифрования.
\end{document}
